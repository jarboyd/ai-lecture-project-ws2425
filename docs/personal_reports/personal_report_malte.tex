% This is samplepaper.tex, a sample chapter demonstrating the
% LLNCS macro package for Springer Computer Science proceedings;
% Version 2.21 of 2022/01/12
%
\documentclass[runningheads]{llncs}
%
\usepackage[german]{babel}
% T1 fonts will be used to generate the final print and online PDFs,
% so please use T1 fonts in your manuscript whenever possible.
% Other font encondings may result in incorrect characters.
%
% Used for displaying a sample figure. If possible, figure files should
% be included in EPS format.
%
% If you use the hyperref package, please uncomment the following two lines
% to display URLs in blue roman font according to Springer's eBook style:
%
\begin{document}
%
\title{Grundlagen KI Projekt persönlicher Report}
%
%\titlerunning{Abbreviated paper title}
% If the paper title is too long for the running head, you can set
% an abbreviated paper title here
%
\author {Malte Elvers}
%
% First names are abbreviated in the running head.
% If there are more than two authors, 'et al.' is used.
%
\institute{Technische Universität Clausthal, Clausthal-Zellerfeld 38678, Deutschland}
%
\maketitle              % typeset the header of the contribution
%

%
%
\section{Individueller Beitrag}
Im Rahmen dieses Projekts habe ich mich mit der Erkennung von Verkehrsschildern mithilfe eines
neuronalen Netzwerks beschäftigt. Mein Ziel war es, ein Modell zu trainieren, das automatisch die
Position von Verkehrsschildern in Bildern erkennt und entsprechende Bounding Boxes sowie den
ausgeschnittenen Bildbereich ausgibt.
Dafür habe ich ein Faster R-CNN-Modell mit PyTorch implementiert, das auf einem ResNet-50-Backbone
basiert.
Neben der Modellerstellung habe ich einen Datensatzloader entwickelt, um die Bilder und die
zugehörigen Bounding-Box-Annotationen aus einzulesen und zu verarbeiten.
Zusätzlich habe ich die Daten transformiert, damit die die Bilder auf eine feste Größe
skaliert werden und gleichzeitig die Bounding Boxes entsprechend anpasst werden.
Anschließend habe ich das Modell trainiert und getestet.
\\
\section{Eigene Arbeitsweise und Reflexion}
Zu Beginn habe ich versucht, das Problem mit einem einfachen CNN und einer direkten
Bounding-Box-Regression zu lösen. Dabei habe ich jedoch festgestellt, dass das Modell keine
zuverlässigen Positionen vorhergesagt hat, sondern stets eine Box in der Bildmitte platziert hat.
Dies lag daran, dass das Modell keine gezielte Objekterkennung durchführen konnte und stattdessen
versuchte, den durchschnittlichen Fehler zu minimieren, was dazu führte, dass es die Box immer
zentral platzierte.
\\
Um dieses Problem zu lösen, habe ich mich für ein Faster R-CNN entschieden, das nicht direkt eine
Bounding Box ausgibt, sondern mit einem Region Proposal Network (RPN) arbeitet, das potenzielle
Objektregionen vorschlägt. Dadurch konnte das Modell gezielt nach Schildern suchen, anstatt die
Boxen immer an den durchschnittlichen "besten Positionen" zu platzieren, unabhängig von den
tatsächlich vorhandenen Schildern. Durch den Einsatz eines ResNet-50-Backbones wurden zudem
automatisch wichtige Bildmerkmale extrahiert, wodurch das Modell präziser Lernen konnte.
\\
Ein weiteres Problem war die lange Trainingszeit auf der CPU. Zu Beginn habe ich das Modell auf der
CPU trainiert, was für jede Iteration extrem viel Zeit in Anspruch genommen hat.
Um das Training zu beschleunigen, habe ich das Modell auf CUDA umgestellt, sodass es auf einer GPU
ausgeführt werden konnte.
Dies führte zu einer erheblichen Reduzierung der Rechenzeit, wodurch ich schneller verschiedene
Varianten testen konnte.
Das umstellen des Modells war sehr schnell möglich, da nur einige Python-Zeilen angepasst werden
mussten, die meiste Zeit bei dieser Teilaufgabe hat die eigentliche Installation des CUDA-Toolkits
in Ansprich genommen.
\section{Persönliche Entwicklung}
Durch dieses Projekt habe ich meine Kenntnisse im Bereich der Objekterkennung vertieft, insbesondere
mit Faster R-CNN. Ich habe gelernt, warum eine einfache Bounding-Box-Regression nicht ausreicht
und warum ein Backbone zur Feature-Extraktion notwendig ist, um präzisere Ergebnisse zu erzielen,
also dass ein neuronales Netz nicht einfach durch viele Trainingsbilder "magisch" Objekte erkennen
kann, sondern dass es eine spezielle Struktur benötigt, um gezielt nach Objekten zu suchen.
\\
Ein wichtiger Lerneffekt war auch die Nutzung der GPU für das Training, da ich gesehen habe,
wie stark sich die Rechenzeit durch die Verwendung von CUDA verkürzen lässt.
Das Projekt hat mir gezeigt, wie wichtig es ist, verschiedene Ansätze zu testen und sich nicht von
Anfang an auf nur einen Lösungsweg festzusetzen, da dieser möglicherweise nicht zum gewünschten
Ergebnis führt.
\end{document}